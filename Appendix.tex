\clearpage
\section{Appendix}

\subsection{智能指针}
\label{sec:SmartPointer}
\texttt{shared\_ptr}使用了引用计数(use count)技术,当复制个
\texttt{shared\_ptr}对象时,被管理的资源并没有被复制,而是增加了引用计
数。当析 构一个\texttt{shared\_ptr}对象时,也不会直接释放被管理的的资
源,而是将引用计数减一。当引用计数为0时,才会真正的释放资源。
\texttt{shared\_ptr}可以方便的共享资源而不必创建多个资源。

\texttt{unique\_ptr}则不同。\texttt{unique\_ptr}独占资源,不能拷贝,只
能移动。移动过后的\texttt{unique\_ptr}实例不再占有资源。当
\texttt{unique\_ptr}被析构时,会释放所持有的资源。
\begin{verbatim}
unique_ptr<T> b = std::move(a);
\end{verbatim}
\texttt{weak\_ptr}可以解决\texttt{shared\_ptr}所持有的资源循环引用问题。
\texttt{weak\_ptr}在指向\texttt{shared\_ptr}时,并不会增加
\texttt{shared\_ptr}的引用计数。所以\texttt{weak\_ptr}并不知道
\texttt{shared\_ptr}所持有的资源是否已经被释放。这就要求在使用
\texttt{weak\_ptr}获取\texttt{shared\_ptr}时需要判断
\texttt{shared\_ptr}是否有效。

\begin{verbatim}
struct Boo;
struct Foo{
    std::shared_ptr<Boo> boo;
};
struct Boo{
    std::shared_ptr<Foo> foo;
};
\end{verbatim}

\texttt{Foo}中有一个智能指针指向\texttt{Goo},而\texttt{Goo}中也有一根
智能指针指向\texttt{Foo},这就是循环引用,我们可以使用
\texttt{weak\_ptr}来解决这个问题。

关于智能指针的一些tips:
\begin{itemize}
\item 智能指针抛开类型T,是线程安全的,因为智能指针底层使用的引用计数是
\texttt{atomic}的原子变量,原子变量在自增自减时是线程安全的,这保证了
多线程读写智能指针时是安全的。
\item 尽可能不要使用裸指针初始化智能指针,因为可能存在同一个裸指针初始
  了多个智能指针,在智能指针析构时会造成资源的多次释放。
\item 不要从智能指针中使用\texttt{get}函数返回裸指针去为另一个智能指针
  赋值,因为如果返回的裸指针被释放了,智能指针持有的资源也失效了,对智
  能指针的操作是未定义的行为。
\item \texttt{shared\_ptr}和\texttt{unique\_ptr}都可以用\texttt{reset}
  函数来释放智能指针,但\texttt{shared\_ptr}没有\texttt{release}函数,
  \texttt{unique\_ptr}的\texttt{release}函数会返回内置指针,并将自身置
  为空。
\item 与\texttt{uniqu\_ptr}不同,删除器不是\texttt{shared\_ptr}类型的
  组成部分。假设,\texttt{shared\_ptr<T> sp2(q,deleter2)},尽管
  \texttt{sp1}和\texttt{sp2}有着不同的删除器,但两者的类型是一致的,都
  可以被放入\texttt{vector<shared\_ptr<T>>}类型的同一容器里。
\item 与\texttt{std::unique\_ptr}不同,自定义删除器不会改变
  \texttt{std::shared\_ptr}的大小。其始终是祼指针大小的两倍。
\item \texttt{shread\_ptr}和\texttt{unique\_ptr}都可以持有数组,但是这
  里需要在\texttt{shared\_ptr}构造时传入\texttt{deleter},用来销毁持有
  的数组,默认删除器不支持数组对象,而\texttt{unique\_ptr}无需此操作,因为\texttt{unique\_ptr}重
  载了\texttt{unique\_ptr(T[])}。\textbf{所有不是\texttt{new}分配内存
    的资源都要给\texttt{shared\_ptr}传递一个删除器。}
\item 不可以使用静态对象初始化智能指针,因为静态对象的生命周期和进程
  一样长,而智能指针的析构的时候会导致静态资源被释放。这会导致未定义的
  行为。
\item 不要将\texttt{this}指针返回给\texttt{shared\_ptr}。当希望将
  \texttt{this}指针托管给\texttt{shared\_ptr}时,类需要继承自
  \texttt{std::enable\_shared\_from\_this},并且从
  \texttt{shared\_from\_this()}中获得\texttt{shared\_ptr}指针,否则有
  double free的风险。 这样做可以的原因是继承自
  \texttt{std::enable\_shared\_from\_this}后,内部数据结构会有shared
  count 和 weak count,每次创建新的\texttt{share\_ptr}都会从这里获取,
  这样就只有一个控制块,无论怎样增加和减少,都只有一份计数。
\item \texttt{weak\_ptr}对弱引用计数的获取,实际上是对
  \texttt{shared\_ptr}的引用计数的获取。弱引用计数最少是1,不会出现0。
\end{itemize}
%%% Local Variables:
%%% mode: latex
%%% TeX-master: "EfficientCpp"
%%% End:
