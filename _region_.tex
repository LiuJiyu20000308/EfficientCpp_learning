\message{ !name(../CppLearning.tex)}\documentclass[a4paper,twoside]{article}
\usepackage{geometry}
\geometry{margin=1cm,vmargin={0pt,1cm}}
\setlength{\topmargin}{-2cm}
\setlength{\paperheight}{23cm}
\setlength{\paperwidth}{18cm}
\setlength{\textheight}{19.6cm}
\setlength{\textwidth}{15cm}
\usepackage{makecell}
%\usepackage{fancyhdr}
\usepackage{siunitx}
\usepackage{amssymb}
\usepackage{indentfirst}
\setlength{\parindent}{0.5em}

\pagenumbering{arabic} 
\usepackage{pdfpages}
% useful packages.

\usepackage{ctex}

\usepackage{color}
\usepackage{multirow}
\usepackage{caption}
\usepackage{mathrsfs}
 \usepackage{amsfonts}
\usepackage{amsmath}
\usepackage{amsthm}
\usepackage{enumerate}
\usepackage{xcolor,graphicx,float,subfigure}
\usepackage{epstopdf}
\usepackage{multicol}
\usepackage{fancyhdr}
\usepackage{layout}
\usepackage{listings}
\usepackage{dsfont}
\lstset{language=Matlab}
\lstset{breaklines}
\lstset{extendedchars=false}
\usepackage[colorlinks,linkcolor=blue]{hyperref}
\usepackage{xcolor}
%\usepackage{cite}
%\usepackage[numbers,sort&compress]{natbib} 
%\setcitestyle{open={},close={}}
%\usepackage{natbibspacing}
%\renewcommand{\refname}{}
\usepackage{anyfontsize}

\usepackage{tikz}
\usetikzlibrary{calc}
\usetikzlibrary{arrows.meta}
\tikzset{
  dot/.style={
    circle, fill=black, inner sep=1pt, outer sep=0pt
  },
  dot label/.style={
    circle, inner sep=0pt, outer sep=1pt
  }
  arrow1/.style = {
    draw = black, thick, -{Latex[length = 4mm, width = 1.5mm]},
  }
}


% -------------------
% Theorem Environments
% -------------------
\theoremstyle{definition}
\newtheorem{thm}{Theorem}[section]
\newtheorem{prop}{Proposition}[section]
\newtheorem{lem}{Lemma}[section]
\newtheorem{coro}{Corollary}[section]

\newtheorem{exm}{Example}[section]
\newtheorem{nex}{Non-Example}[section]
\newtheorem{defn}{Definition}[section]
\theoremstyle{remark}
\newtheorem{rmk}{Remark}[section] 
\numberwithin{equation}{section}


\newcommand{\dif}{\mathrm{d}}
\newcommand{\avg}[1]{\left\langle #1 \right\rangle}
\newcommand{\difFrac}[2]{\frac{\dif #1}{\dif #2}}
\newcommand{\pdfFrac}[2]{\frac{\partial #1}{\partial #2}}
\newcommand{\OFL}{\mathrm{OFL}}
\newcommand{\UFL}{\mathrm{UFL}}
\newcommand{\fl}{\mathrm{fl}}
\newcommand{\op}{\odot}
\newcommand{\Eabs}{E_{\mathrm{abs}}}
\newcommand{\Erel}{E_{\mathrm{rel}}}

\newcommand{\Zero}{\hat{0}}
\newcommand{\One}{\hat{1}}
\newcommand{\Int}{\mathrm{int}}
\newcommand{\unitV}{\mathds{1}}

\newcommand{\bmi}{\mathbf{i}}
\newcommand{\bmj}{\mathbf{j}}
\newcommand{\bmn}{\mathbf{n}}

\newcommand{\dist}[2]{\text{dist}\left(#1, #2\right)}
\newcommand{\scientific}[2]{#1 \times 10^{#2}}


%\newcommand{\Dim}{{\mathbf{D}}}
\newcommand{\Dim}{{\scriptsize \textsf{D}}}
\newcommand{\me}{\mathrm{e}}
\newcommand{\mi}{\mathrm{i}}

%\newcommand{\mod}{\mathrm{mod}}
\newcommand{\curve}[1]{\widetilde{#1}}
%\newcommand{\dt}{\delta t}
\newcommand{\dt}{\tau}
\newcommand{\isCovered}{\mathbin{ < \! \! \! \! \cdot }}
%\newcommand{\cIncluded}{\mathbin{ \prec \! \! \! \cdot }}
\newcommand{\coveredBy}{\lhd}
%\newcommand{\regrz}[1]{\mathrm{cl}\left(\mathrm{int}\left(#1\right)\right)}
\newcommand{\regrz}[1]{\mathrm{reg}\left(#1\right)}
%\newcommand{\sgncup}{\ \hat{\cup} \ }
\newcommand{\Span}{\mathrm{span}}
\newcommand{\timeline}[2]{\phi_{t_0}^{#1}\left( #2 \right)}
\newcommand{\timeBP}[1]{\overleftarrow{#1}}
\newcommand{\timeBPA}[1]{\mathring{\overleftarrow{#1}}}
\newcommand{\streak}[2]{\Psi_{t_0}^{#1}\left(#2\right)}
\newcommand{\timelineA}[2]{\mathring{\phi}_{t_0,#2}^{#1}}
\newcommand{\DRLN}[1]{{\cal D}_{\curve{#1}}}
\newcommand{\DRLLN}[1]{{\cal D}_{\overline{#1}}}
\newcommand{\DRLNA}[1]{\mathring{\cal D}_{\curve{#1}}}
%\newcommand{\oplusDR}{\,\overline{\oplus}\,}
\newcommand{\oplusDR}{\,\bar{\oplus}\,}
\newcommand{\qo}{\hat{q}}
\newcommand{\xo}{\hat{x}}
\newcommand{\yo}{\hat{y}}
\newcommand{\closure}[1]{\textrm{cl}\left(#1\right)}
\newcommand{\vertexSequence}[4]{
  \left( #1 \rightarrow #2 \rightarrow #3 \rightarrow #4 \rightarrow #1\right)}

\newcommand{\ppSpace}{\Pi_{<\kappa,\bm{\xi},\bm{\nu}}}
\newcommand{\pnSpace}{\mathbb{P}_{<\kappa}}
\newcommand{\pnSpaceK}[1]{\mathbb{P}_{#1}}

\newcommand{\Pyr}[2]{\textrm{Pyr}_{\cal{#1}}\left(\mathbf{#2}\right)}

\let\OldTexttt\texttt
\renewcommand{\texttt}[1]{{\color{blue} \OldTexttt{#1}}}

%\pagestyle{plain}
\pagestyle{fancy}
\fancyhf{}
\fancyhead[LE,RO]{\textbf{\thepage}}

\makeatletter
\newcommand\sixteen{\@setfontsize\sixteen{17pt}{6}}
\renewcommand{\maketitle}{\bgroup\setlength{\parindent}{0pt}
\begin{flushleft}
\sixteen\bfseries \@title
\medskip
\end{flushleft}
\textit{\@author}
\egroup}
\makeatother

\title{C++ Learning}
\begin{document}

\message{ !name(sec/EffectiveModernCpp.tex) !offset(-167) }
\newpage

\section{类型推导}

\subsection{理解模板类型推导}
\label{sec:Item2-1}
\begin{verbatim}
template<typename T>
void f(ParamType param);
f(expr);
\end{verbatim}

在编译期间,编译器会用\texttt{expr}推到两个类型:\texttt{T}以及
\texttt{ParamType}。这两种类型通常是不一样的,因为\texttt{ParamType}经
常带修饰符,例如\texttt{const}或者\texttt{reference}。\texttt{T}类型的
推导不仅仅取决于\texttt{expr},也取决于\texttt{ParamType}的类型,分为
三种情况:
\begin{itemize}
\item \texttt{ParamType}是指针或者引用类型,但不是通用引用
  (Universal reference,\texttt{ParamType}带有\texttt{\&\&}修饰符,见
  Section~\ref{sec:ForwardingRef}).
\item \texttt{ParamType}是通用引用.
\item \texttt{ParamType}既不是指针也不是引用。
\end{itemize}

以下将分类进行讨论:
\begin{itemize}
\item 当\texttt{ParamType}是指针或者引用类型,但不是通用引用时,类型推
  导原理为:
    \begin{enumerate}
    \item 如果\texttt{expr}类型是引用的话,忽略引用。
    \item 通过令\texttt{expr}
    \end{enumerate}
    和\texttt{ParamType}类型一致来决定\texttt{T}。
  例如:
\begin{verbatim}
template<typename T>
void f(T& param);
void g(const T& param); // param is now a ref-to-const

int x = 27;        // x is an int
const int cx = x;  // cx is a const int
const int& rx = x; // rx is a reference to x as a const int

f(x);  // T is int, param's type is int&
f(cx); // T is const int, param's type is const int&
f(rx); // T is const int, param's type is const int&

g(x); // T is int, param's type is const int&
g(cx); // T is int, param's type is const int&
g(rx); // T is int, param's type is const int&
\end{verbatim}

  
\item \texttt{ParamType}为通用引用时,类型推导(见
  Section~\ref{sec:Item2-24})如下:
  \begin{enumerate}
  \item 如果\texttt{expr}为左值,则\texttt{T}和\texttt{ParamType}均为
    左值引用,\textbf{这是唯一一种\texttt{T}被推导为引用的情况},另外
    值得注意的是,虽然\texttt{ParamType}带有\texttt{\&\&},\textbf{但
      还是被推导为左值引用。}
  \item 如果\texttt{expr}为右值,那么推导和第一种情况类似。
  \end{enumerate}

  例如:
\begin{verbatim}
template<typename T>
void f(T&& param); // param is now a universal reference

int x = 27; // as before
const int cx = x; // as before
const int& rx = x; // as before

f(x);  // x is lvalue, so T is int&,param's type is also int&
f(cx); // cx is lvalue, so T is const int&, param's type is also const int&
f(rx); // rx is lvalue, so T is const int&, param's type is also const int&
f(27); // 27 is rvalue, so T is int, param's type is therefore int&&
\end{verbatim}

\item \texttt{ParamType}既不是指针也不是引用时,这意味着\texttt{param}
  将会传递输入参数的副本,因此此时的类型推导如下:
  \begin{enumerate}
  \item 如果\texttt{expr}是一个引用,则忽略引用部分。
  \item 如果忽略引用之后是\texttt{const}的,也将\texttt{const}忽略。如
    果是\texttt{volatile}(见Section~\ref{sec:Item2-40})的也要忽略。
  \end{enumerate}
  例如:
\begin{verbatim}
template<typename T>
void f(T param); // param is now passed by value

int x = 27;       
const int cx = x; 
const int& rx = x; 
f(x);  // T's and param's types are both int
f(cx); // T's and param's types are again both int
f(rx); // T's and param's types are still both int
\end{verbatim}
  这样做的原因是\texttt{cx}和\texttt{rx}不能被更改不代表\texttt{param}
  不能被更改,另外,要意识到\textbf{\texttt{const}和\texttt{volatile}
    只有在按值传入参数的时候会被忽略。}

  如果\texttt{expr}是一个指向\texttt{const}对象的\texttt{const}指针的
  话,\textbf{指针的\texttt{const}会被忽略}:
\begin{verbatim}
// ptr is const pointer to const object
const char* const ptr = "Fun with pointers";

// type deduced for param will be const char*
f(ptr);
\end{verbatim}
\end{itemize}

\textbf{在许多情况下,数组会退化为指向第一个元素的指针。当数组被传入到
  一个by-value参数的模板中呢?}

\begin{verbatim}
template<typename T>
void f(T param); // template with by-value parameter

const char name[] = "J. P. Briggs"; // name's type is const char[13]
f(name); // name is array, but T deduced as const char*
\end{verbatim}

但是如果\texttt{expr}是一个引用呢?

\begin{verbatim}
template<typename T>
void f(T& param); // template with by-reference parameter

f(name); // T is deduced to be const char [13]
//paramType is const char (&)[13]
\end{verbatim}

这种语法使得我们可以利用模板来推导出数组元素个数:
\begin{verbatim}
template<typename T, std::size_t N> 
constexpr std::size_t arraySize(T (&)[N]) noexcept{
  return N; 
}
\end{verbatim}

将函数声明为\texttt{constexpr}可以让结果在编译器就可以使用,同时也能让
编译器生成更好的代码(见Section \ref{sec:Item2-14}, \ref{sec:Item2-15}):
\begin{verbatim}
int keyVals[] = { 1, 3, 7, 9, 11, 22, 35 }; // keyVals has 7 elements
int mappedVals[arraySize(keyVals)]; // so does mappedVals
std::array<int, arraySize(keyVals)> mappedVals; // mappedVals'size is 7 
\end{verbatim}

同样的,\textbf{函数类型也会退化到function指针}:

\begin{verbatim}
void someFunc(int, double); // someFunc is a function with type
void(int, double)

template<typename T>
void f1(T param); // in f1, param passed by value
template<typename T>
void f2(T& param); // in f2, param passed by ref

f1(someFunc); // param deduced as ptr-to-func;
// type is void (*)(int, double)
f2(someFunc); // param deduced as ref-to-func;
// type is void (&)(int, double)
\end{verbatim}

\subsection{理解auto类型推导}
\label{sec:Item2-2}

当一个变量用\texttt{auto}声明的时候,\texttt{auto}扮演模板中的
\texttt{T}的作用,变量的类型修饰符可以看做\texttt{ParamType}。

同样的,\texttt{auto}也分为三个情况:
\begin{itemize}
\item 类型修饰符是指针或引用,但不是通用引用。
\item 类型修饰符是通用引用。
\item 类型修饰符既不是指针也不是引用。
\end{itemize}

\begin{verbatim}
auto x = 27;        // case 3 (x is neither ptr nor ref)
const auto cx = x;  // case 3 (cx isn't either)
const auto& rx = x; // case 1 (rx is a non-universal ref)

auto&& uref1 = x;  // uref1's type is int&
auto&& uref2 = cx; // uref2's type is const int&
auto&& uref3 = 27; // uref3's type is int&&

const char name[] = "R. N. Briggs"; // name's type is const char[13]
auto arr1 = name; // arr1's type is const char*
auto& arr2 = name; // arr2's type is const char (&)[13]

void someFunc(int, double); //void(int, double)
auto func1 = someFunc; // func1's type is void (*)(int, double)
auto& func2 = someFunc; // func2's type is void (&)(int, double)
\end{verbatim}
可以看到,\texttt{auto}类型推导和模板类型推导一致。只有一种情况下,它
们是不同的:

\begin{verbatim}
int x1 = 27;
int x2(27);
int x3 = { 27 };
int x4{ 27 };
\end{verbatim}
它们都是一个结果:值为27的\texttt{int},但是如果我们换成\texttt{auto}
的话,对于后两个而言,会定义一个只包含一个元素27的类型为
\texttt{std::initializer\_list<int>}的对象。
\begin{verbatim}
auto x1 = 27; // type is int, value is 27
auto x2(27); // ditto
auto x3 = { 27 }; // type is std::initializer_list<int>, value is { 27 }
auto x4{ 27 }; // ditto
\end{verbatim}

花括号初始化的时候,\texttt{auto}会推断为
\texttt{std::initializer\_list<int>},假如花括号内部类型不相同,则会报
错:
\begin{verbatim}
auto x5 = { 1, 2, 3.0 }; // error! can't deduce T for std::initializer_list<T>
\end{verbatim}

然而对于模板而言,使用花括号的话类型推导会失败:
\begin{verbatim}
auto x = { 11, 23, 9 };
template<typename T> 
void f(T param);
f({ 11, 23, 9 }); // error! can't deduce type for T

template<typename T>
void f(std::initializer_list<T> initList);
f({ 11, 23, 9 });
\end{verbatim}

C++14中\texttt{auto}可以用在函数返回类型中,也可以在lambda表达式的参数
声明中使用,然而\textbf{它们使用的是模板类型推导而不是\texttt{auto}类
  型推导。}

\begin{verbatim}
auto createInitList(){
  return { 1, 2, 3 }; // error: can't deduce type.
}

std::vector<int> v;
auto resetV = [&v](const auto& newValue) { v = newValue; }; 
resetV({ 1, 2, 3 }); // error! can't deduce type.
\end{verbatim}


\subsection{理解decltype}
\label{sec:Item2-3}

\texttt{decltype}会告诉你表达式的类型:
\begin{verbatim}
const int i = 0; // decltype(i) is const int
bool f(const Widget& w); // decltype(w) is const Widget&
// decltype(f) is bool(const Widget&)

struct Point {
int x, y; // decltype(Point::x) is int
}; // decltype(Point::y) is int
Widget w; // decltype(w) is Widget

if (f(w)) … // decltype(f(w)) is bool

template<typename T> // simplified version of std::vector
class vector {
public:
…
T& operator[](std::size_t index);
…
};
vector<int> v; // decltype(v) is vector<int>
if (v[0] == 0) … // decltype(v[0]) is int&
\end{verbatim}

在C++11中\texttt{decltype}主要用来声明那些返回值依赖于参数类型的函数模
板。例如,假如函数要返回\texttt{vector<bool>}中的一个元素,然而
\texttt{operator[]}并不会返回\texttt{bool\&}而是一个新的object,这里的
关键是返回值是依赖于容器的:
\begin{verbatim}
// works, but requires refinement.
template<typename Container, typename Index> 
auto authAndAccess(Container& c, Index i) -> decltype(c[i]){
  authenticateUser();
  return c[i];
}
\end{verbatim}
这就是C++11中的\textbf{返回值类型后置语法}。在C++14中可以只留下
\texttt{auto}:
\begin{verbatim}
template<typename Container, typename Index>
auto authAndAccess(Container& c, Index i){
  authenticateUser();
  return c[i]; // return type deduced from c[i]
}
\end{verbatim}

这里\texttt{auto}按照模板类型推导,然而这里是存在问题的,虽然
\texttt{operator[]}返回\texttt{T\&},但是在推导时引用会被忽略,因此下
面的代码是无法编译的:
\begin{verbatim}
std::deque<int> d;
authAndAccess(d, 5) = 10; 
\end{verbatim}

C++14中,我们可以这样改进,我们称之为\texttt{decltype}推导方法:
\begin{verbatim}
// still have refinement.
template<typename Container, typename Index> 
decltype(auto) authAndAccess(Container& c, Index i){
  authenticateUser();
  return c[i];
} // return type T&.
\end{verbatim}

这种推导方式也可以用作初始化表达式:
\begin{verbatim}
Widget w;
const Widget& cw = w;
auto myWidget1 = cw; // myWidget1's type is Widget
decltype(auto) myWidget2 = cw; // myWidget2's type is const Widget&
\end{verbatim}

上面说到的refinement实际上是说,如果我想传入一个右值,例如一个临时的
Container,这是无法编译通过的,因此我们还可以做出改进:

\begin{verbatim}
template<typename Container, typename Index> // final C++14 version
decltype(auto) authAndAccess(Container&& c, Index i){
  authenticateUser();
  return std::forward<Container>(c)[i];
}

template<typename Container, typename Index> // final C++11 version
auto authAndAccess(Container&& c, Index i)
-> decltype(std::forward<Container>(c)[i]){
  authenticateUser();
  return std::forward<Container>(c)[i];
}
\end{verbatim}
这里我们使用了\texttt{std::forward},参见
Section~\ref{sec:ForwardingRef}.

当然,在极少数情况下\texttt{decltype}会得不到你期望的类型:如果对一个
名称使用\texttt{decltype}是没有问题的,但是如果对于一个比名称复杂的左
值表达式,\texttt{decltype}都会返回左值引用:
\begin{verbatim}
decltype(auto) f1(){
  int x = 0;
  return x; // decltype(x) is int, so f1 returns int
}
decltype(auto) f2(){
  int x = 0;
  return (x); // decltype((x)) is int&, so f2 returns int&
}
\end{verbatim}

\subsection{了解如何去看类型推导}
\label{sec:Item2-4}

有以下几种方法:
\begin{itemize}
\item 通过IDE编辑器来看。
\item 通过编译器诊断:
\begin{verbatim}
template<typename T> // declaration only for TD;
class TD; // TD == "Type Displayer"
TD<decltype(x)> xType;
\end{verbatim}
  编译器会报错:
\begin{verbatim}
error: 'xType' uses undefined class 'TD<int>'
\end{verbatim}
\item 运行期也可以输出:
\begin{verbatim}
#include <boost/type_index.hpp>
std::cout << typeid(x).name() << '\n'; // display types for x
using boost::typeindex::type_id_with_cvr;
cout << "T = " << type_id_with_cvr<T>().pretty_name();
\end{verbatim}
\end{itemize}

\section{auto}

\subsection{相比于显式类型声明,更倾向于用auto}
\label{sec:Item2-5}

\begin{itemize}
\item 使用\texttt{auto}可以避免忘记初始化:
\begin{verbatim}
int x1; // potentially uninitialized
auto x2; // error! initializer required
\end{verbatim}
\item 也可以在声明与迭代器相关的局部变量时省点笔墨:
\begin{verbatim}
template<typename It> // as before
void dwim(It b, It e){
  while (b != e) {
    // typename std::iterator_traits<It>::value_type
    auto currValue = *b; 
  }
}
\end{verbatim}
\item 也可以表示那些只有编译器才知道的类型:
\begin{verbatim}
auto derefUPLess =  [](const std::unique_ptr<Widget>& p1,
                       const std::unique_ptr<Widget>& p2)
{ return *p1 < *p2; };
\end{verbatim}
  在C++14中这种特性得到了延续:
\begin{verbatim}
auto derefUPLess =  [](const auto& p1, const auto& p2)
{ return *p1 < *p2; };
\end{verbatim}
\end{itemize}

上边的函数的类型也可以写作:
\begin{verbatim}
std::function<bool(const std::unique_ptr<Widget>&,
                   const std::unique_ptr<Widget>&)>
derefUPLess =  [](const std::unique_ptr<Widget>& p1,
                       const std::unique_ptr<Widget>& p2)
{ return *p1 < *p2; };;
\end{verbatim}
它可以用来表示任何可调用的object。

但是这样的话即使抛开语法的冗长,也需要重复参数类型,另外,使用
\texttt{std::function}与使用\texttt{auto} 不同: 一个
持有闭包的\texttt{auto}声明变量与closure具有相同的类型,因此它只使用
closure所需的内存。 持有closure的\texttt{std::function}声明变量的类型
是\texttt{std::function}  object的实例化模板,通常会比前者占用更多的内
存,并且通过\texttt{std::function}调用一个closure会更慢。

另外,\texttt{auto}也可以避免 "type shortcuts":
\begin{verbatim}
std::unordered_map<std::string, int> m;
for (const std::pair<std::string, int>& p : m){
  // do something with p
}
\end{verbatim}
这看起来很正确,但是却存在问题:\texttt{std::unordered\_map}的key类型
是\texttt{const}的,所以Hash表中的\texttt{std::pair}的类型是
\texttt{std::pair<const std::string, int>}。因此,编译器会创造一个临时
object绑定在p上,之后再销毁。所以为了避免这种意外情况,我们可以:
\begin{verbatim}
for (const auto& p : m){
  // as before
}
\end{verbatim}

当然,\texttt{auto}也不是完美的,有的时候类型推断会得到意料之外的结果,
可以看Section~\ref{sec:Item2-2},\ref{sec:Item2-6}. 另外,在编程中,有
时如果为了可读性更好,也可以少使用一点\texttt{auto}。

\subsection{当auto推断出意料之外的类型时请使用显式类型初始化}
\label{sec:Item2-6}

有些时候\texttt{auto}类型推导与你期望的结果不同,例如:如果我有一个输
入\texttt{Widget}输出\texttt{std::vector<bool>}的函数:
\begin{verbatim}
std::vector<bool> features(const Widget& w);
Widget w;
bool highPriority = features(w)[5];
processWidget(w, highPriority);
\end{verbatim}
这样看起来没有问题,但如果我们换成\texttt{auto}类型声明呢?
\begin{verbatim}
auto highPriority = features(w)[5];
processWidget(w, highPriority); // undefined behavior!
\end{verbatim}

\textbf{此时\texttt{highPriority}的类型不再是\texttt{bool}而是
\texttt{std::vector<bool>::reference}了。}因为
\texttt{std::vector<bool>}并不是容器,而是一个压缩形式的\texttt{bool},
一个字节表示一个\texttt{bool},然而C++不允许引用一个字节,因此有了
\texttt{std::vector<bool>::reference}像一个\texttt{bool\&},为了表现像
一个\texttt{bool\&},它存在向\texttt{bool}的隐式转换。

此时\texttt{highPriority}指向了一个临时变量所占据的一个字节,在语句过
后,\texttt{features(w)}这个临时变量销毁,\texttt{highPriority}相当于
变成了一个悬挂指针,因此后续会出现undefined behavior。

\texttt{std::vector<bool>::reference}是一种\textbf{代理类}:为了模拟和
增强某些其他类型的行为而存在的类。代理类被用于多种用途:
\begin{itemize}
\item \texttt{std::vector<bool>::reference}是用于提供
  \texttt{std::vector<bool>}的\texttt{operator[]}操作返回一个指向字节
  的引用的幻觉的。
\item 智能指针用于资源管理。
\item 表达式模板。
\end{itemize}

\section{Modern C++}

\subsection{构造object时注意区分()和\{\}}
\label{sec:Item2-7}

初始化分为三种:括号、等号或者大括号,有时候还可能等号和大括号同时
出现:
\begin{verbatim}
int x(0); // initializer is in parentheses
int y = 0; // initializer follows "="
int z{ 0 }; // initializer is in braces
\end{verbatim}

对于用户定义的类而言,区分赋值和初始化是很重要的,因为会调用不同的函数:
\begin{verbatim}
Widget w1; // call default constructor
Widget w2 = w1; // not an assignment; calls copy ctor
w1 = w2; // an assignment; calls copy operator=
\end{verbatim}

C++98无法直接初始化一个具有特定值的STL容器,但是C++11有了通用初始化
(uniform initialization)就可以了:
\begin{verbatim}
std::vector<int> v{ 1, 3, 5 };
\end{verbatim}

大括号也可以用来为non-static数据成员提供默认初始值:
\begin{verbatim}
class Widget {
private:
  int x{ 0 }; // fine, x's default value is 0
  int y = 0; // also fine
  int z(0); // error!
};
\end{verbatim}

不可拷贝的对象(例如\texttt{std::atomics},见
Section~\ref{sec:Item2-10})也可以使用大括号或者括号来初始化,但不能用
等号:
\begin{verbatim}
std::atomic<int> ai1{ 0 }; // fine
std::atomic<int> ai2(0);   // fine
std::atomic<int> ai3 = 0;  // error!
\end{verbatim}

另外,大括号初始化的一个新颖的特点就是

%%% Local Variables:
%%% mode: latex
%%% TeX-master: "../CppLearning"
%%% End:

\message{ !name(../CppLearning.tex) !offset(-540) }

\end{document}
%%% Local Variables:
%%% mode: latex
%%% TeX-master: t
%%% End:
