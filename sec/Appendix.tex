\clearpage
\section{Appendix}

\subsection{三五法则}
\label{sec:ThreeFive}

\begin{itemize}
\item 需要析构函数的类也需要拷贝构造函数和拷贝赋值函数。
\item  需要拷贝操作的类也需要赋值操作,反之亦然。需要拷贝操作代表这个
  类在拷贝时需要进行一些额外的操作。赋值操作=先析构+拷贝,所以拷贝需要
  的赋值也需要。反之亦然。
\item 析构函数不能是私有的或删除的,否则成员便无法销毁。
\item 如果一个类有私有的或不可访问的析构函数,那么其默认和拷贝构造函数
  会被定义为删除的。
\item 如果一个类有\texttt{const}或引用成员,则不能使用默认的拷贝赋值操
  作。 
\end{itemize}
\subsection{智能指针}
\label{sec:SmartPointer}
\texttt{shared\_ptr}使用了引用计数(use count)技术,当复制个
\texttt{shared\_ptr}对象时,被管理的资源并没有被复制,而是增加了引用计
数。当析 构一个\texttt{shared\_ptr}对象时,也不会直接释放被管理的的资
源,而是将引用计数减一。当引用计数为0时,才会真正的释放资源。
\texttt{shared\_ptr}可以方便的共享资源而不必创建多个资源。

\texttt{unique\_ptr}则不同。\texttt{unique\_ptr}独占资源,不能拷贝,只
能移动。移动过后的\texttt{unique\_ptr}实例不再占有资源。当
\texttt{unique\_ptr}被析构时,会释放所持有的资源。
\begin{verbatim}
unique_ptr<T> b = std::move(a);
\end{verbatim}
\texttt{weak\_ptr}可以解决\texttt{shared\_ptr}所持有的资源循环引用问题。
\texttt{weak\_ptr}在指向\texttt{shared\_ptr}时,并不会增加
\texttt{shared\_ptr}的引用计数。所以\texttt{weak\_ptr}并不知道
\texttt{shared\_ptr}所持有的资源是否已经被释放。这就要求在使用
\texttt{weak\_ptr}获取\texttt{shared\_ptr}时需要判断
\texttt{shared\_ptr}是否有效。

\begin{verbatim}
struct Boo;
struct Foo{
    std::shared_ptr<Boo> boo;
};
struct Boo{
    std::shared_ptr<Foo> foo;
};
\end{verbatim}

\texttt{Foo}中有一个智能指针指向\texttt{Goo},而\texttt{Goo}中也有一根
智能指针指向\texttt{Foo},这就是循环引用,我们可以使用
\texttt{weak\_ptr}来解决这个问题。

关于智能指针的一些tips:
\begin{itemize}
\item 智能指针抛开类型T,是线程安全的,因为智能指针底层使用的引用计数是
\texttt{atomic}的原子变量,原子变量在自增自减时是线程安全的,这保证了
多线程读写智能指针时是安全的。
\item 尽可能不要使用裸指针初始化智能指针,因为可能存在同一个裸指针初始
  了多个智能指针,在智能指针析构时会造成资源的多次释放。
\item 不要从智能指针中使用\texttt{get}函数返回裸指针去为另一个智能指针
  赋值,因为如果返回的裸指针被释放了,智能指针持有的资源也失效了,对智
  能指针的操作是未定义的行为。
\item \texttt{shared\_ptr}和\texttt{unique\_ptr}都可以用\texttt{reset}
  函数来释放智能指针,但\texttt{shared\_ptr}没有\texttt{release}函数,
  \texttt{unique\_ptr}的\texttt{release}函数会返回内置指针,并将自身置
  为空。
\item 与\texttt{uniqu\_ptr}不同,删除器不是\texttt{shared\_ptr}类型的
  组成部分。假设,\texttt{shared\_ptr<T> sp2(q,deleter2)},尽管
  \texttt{sp1}和\texttt{sp2}有着不同的删除器,但两者的类型是一致的,都
  可以被放入\texttt{vector<shared\_ptr<T>>}类型的同一容器里。
\item 与\texttt{std::unique\_ptr}不同,自定义删除器不会改变
  \texttt{std::shared\_ptr}的大小。其始终是祼指针大小的两倍。
\item \texttt{shread\_ptr}和\texttt{unique\_ptr}都可以持有数组,但是这
  里需要在\texttt{shared\_ptr}构造时传入\texttt{deleter},用来销毁持有
  的数组,默认删除器不支持数组对象,而\texttt{unique\_ptr}无需此操作,因为\texttt{unique\_ptr}重
  载了\texttt{unique\_ptr(T[])}。\textbf{所有不是\texttt{new}分配内存
    的资源都要给\texttt{shared\_ptr}传递一个删除器。}
\item 不可以使用静态对象初始化智能指针,因为静态对象的生命周期和进程
  一样长,而智能指针的析构的时候会导致静态资源被释放。这会导致未定义的
  行为。
\item 不要将\texttt{this}指针返回给\texttt{shared\_ptr}。当希望将
  \texttt{this}指针托管给\texttt{shared\_ptr}时,类需要继承自
  \texttt{std::enable\_shared\_from\_this},并且从
  \texttt{shared\_from\_this()}中获得\texttt{shared\_ptr}指针,否则有
  double free的风险。 这样做可以的原因是继承自
  \texttt{std::enable\_shared\_from\_this}后,内部数据结构会有shared
  count 和 weak count,每次创建新的\texttt{share\_ptr}都会从这里获取,
  这样就只有一个控制块,无论怎样增加和减少,都只有一份计数。
\item \texttt{weak\_ptr}对弱引用计数的获取,实际上是对
  \texttt{shared\_ptr}的引用计数的获取。弱引用计数最少是1,不会出现0。
\end{itemize}

\subsection{万能引用}
\label{sec:ForwardingRef}

首先理解引用折叠,引用折叠会发生在模板实例化、\texttt{auto}类型推导、
创建和运用\texttt{typedef}和别名声明、以及\texttt{decltype}语境中。

引用折叠规则:
\begin{itemize}
\item 所有右值引用折叠到右值引用上仍然是一个右值引用。如\texttt{X\&\&
    \&\&}折叠为\texttt{X\&\&}。 
\item 所有的其他引用类型之间的折叠都将变成左值引用。如\texttt{X\& \&},
  \texttt{X\& \&\&}, \texttt{X\&\& \&}折叠为\texttt{X\&}。可见左值引用
  会传染,沾上一个左值引用就变左值引用了。根本原因:在一处声明为左值,
  就说明该对象为持久对象,编译器就必须保证此对象可靠(左值)。
\end{itemize}
\begin{itemize}
\item 当万能引用(\texttt{T\&\& param})绑定到左值时,由于万能引用也是
  一个引用,而左值只能绑定到左值引用。因此,\texttt{T}会被推导为
  \texttt{T\&}类型。从而\texttt{param}的类型为\texttt{T\& \&\&},引用
  折叠后的类型为\texttt{T\&}。 
\item 当万能引用(\texttt{T\&\& param})绑定到右值时,同理,右值只能绑
  定到右值引用上,故\texttt{T}会被推导为\texttt{T}类型。从而param的类
  型就是\texttt{T\&\&}(右值引用)。
\end{itemize}
\subsection{std::function}
\label{sec:function}


\subsection{std::bind}
\label{sec:bind}

 
\subsection{表达式模板}
\label{sec:ExpreesionTemp}


\subsection{重载、重写、重定义}
\label{sec:Overload}
\begin{itemize}
\item \textbf{重载}(overload):函数名相同,函数的参数个数、参数类型
  或参数顺序三者中必须至少有一种不同。函数返回值的类型可以相同,也可以
  不相同。发生在一个类内部。 \texttt{const}关键字可以用于对重载函数的
  区分。 
\item \textbf{重定义}:也叫做隐藏,子类重新定义父类中有相同名称的非虚
  函数 ( 参数列表可以不同 ) ,指派生类的函数屏蔽了与其同名的基类函数。
  发生在继承中。 
\item \textbf{重写}(override):也叫做覆盖,一般发生在子类和父类继承
  关系之间。子类重新定义父类中有相同名称和参数的虚函数。
\end{itemize}

重写需要注意:
\begin{itemize}
\item 被重写的函数不能是static的。必须是virtual的
\item 重写函数必须有相同的类型,名称和参数列表
\item 重写函数的访问修饰符可以不同。尽管virtual是private的,派生类中重
  写改写为public,protected也是可以的。
\end{itemize}

重定义规则如下:
\begin{itemize}
\item 如果派生类的函数和基类的函数同名,但是参数不同,此时,不管有无
  virtual,基类的函数被隐藏。
\item 如果派生类的函数与基类的函数同名,并且参数也相同,但是基类函数没
  有vitual关键字,此时,基类的函数被隐藏(如果相同有virtual就是重写覆
  盖了)。 
\end{itemize}

重定义存在的意义是避免基类修改某些实现导致继承类正常运行的代码出现异常
行为。

但是如果upcast,即用基类指针指向继承类object,就只能调用基类的函数。

如果想继续使用基类的函数,需要在继承类中加入 \texttt{using Base::f}。
\begin{verbatim}
class Base {
public:
  virtual int f() const {}
  virtual void f(string) const {}
  virtual void g() const {}
};

class Derived2 : public Base {
public:
  // Overriding a virtual function:
  // string version has been hidden!
  int f() const {
    cout << "Derived2::f()\n";
    return 2;
  }
};

class Derived3 : public Base {
public:
  // Cannot change return type:
  //void f() const{ cout << "Derived3::f()\n";}
};

class Derived4 : public Base {
public:
  // Change argument list:
  // two base::f have been hidden
  int f(int) const {
    cout << "Derived4::f()\n";
    return 4;
  }
};

Derived4 d4;
x = d4.f() // error!
Base *base = new Derived4;
base -> f(1) // error! Derived version is unavailable.
br -> f();
br -> f("string"); // Base version available.
\end{verbatim}

\subsection{virtual实现原理}
\label{sec:vptr}
虚函数实现多态的机制,严格来说是动态多态,是在出现运行的时候实现的。

虚函数的实现原理:每个虚函数都会有一个与之对应的虚函数表,该虚函数表的
实质是一个指针数组,存放的是每一个对象的虚函数入口地址。对于一个派生类
来说,他会继承基类的虚函数表同时增加自己的虚函数入口地址,如果派生类重
写了基类的虚函数的话,那么继承过来的虚函数入口地址将被派生类的重写虚函
数入口地址替代。那么在程序运行时会发生动态绑定,将父类指针绑定到实例化
的对象实现多态。 

\subsection{零碎知识点}

\begin{itemize}
\item 普通左值引用无法指向右值,但常量左值引用可以指向右值,常用于拷贝
  构造函数。
\item 
\end{itemize}
%%% Local Variables:
%%% mode: latex
%%% TeX-master: "../CppLearning"
%%% End:
